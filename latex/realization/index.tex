
\chapter{Realization}
\label{sec:realization}

\section{Architecture}
The components of Histo described in Chapter \ref{sec:histo} all describe self-contained functionality and we therefore implement each of them as a separate module.

At the highest level is the Histo module which only implements the history tracking functionality described in Section \ref{sec:histo.committing}.
All other features are still exposed in the Histo module but delegated to specialized modules:

\begin{itemize}
\item \emph{histo-sync}: Implements the synchronization protocol of Section \ref{sec:histo.protocol}
\item \emph{lca}: Implements a lowest-common ancestor algorithm described in \ref{sec:background.lca}
\item \emph{graph-diff}: Implements the commit history difference algorithm of Section \ref{sec:histo.graph-diff}
\item \emph{content-addressable}: Implements a content-addressable store interface described in Section \ref{sec:background.cas}
\item \emph{diff-patch-merge}: Implements the three-way-merge algorithms described in Section \ref{sec:histo.merging}
\end{itemize}

Through this modular approach we enable the support of alternative data structures and other history-based synchronization protocols.

\section{Technology}
The use of Web Standards is the most common option to implement application logic that can run unmodified on a large number of platforms.
Desktop computers, notebooks, smartphones and tablets all come with implementations of web standards.
JavaScript has become the only high-level programming language that can run on any modern consumer device.
In recent years JavaScript is even increasingly used on server platforms to implement application backends.
Node.js has become the most popular server-side execution environment for JavaScript.\\

As described in Section \ref{sec:background.html5}, modern browsers come with support for client-side databases in the form of IndexedDB.
The IndexedDB implementation of the Chrome Browser is based on LevelDB, which is a key-value store developed by Google \cite{leveldb}.
LevelDB is inspired by Google's BigTable \cite{chang2008bigtable} and was open sourced in 2011.
The Node.js community made LevelDB usable from JavaScript in the form of the LevelDOWN module \cite{leveldown}.
The same interface used by LevelDOWN is available as level.js \cite{leveljs}, which wraps IndexedDB in the browser.
This means we have the same database API available both in browsers and on the server using Node.js.\\
We therefore decided to base the implementation of Histo on JavaScript, using the LevelDOWN interface for the storage layer.

\section{Correctness and Robustness}
We implement the Histo modules in a test-driven manner.
This ensures that all implementations are continuously checked for correctness as we progress.
The implementations and test suites of all modules are attached to the thesis and can be found online where they continue to be improved \cite{histo-source}.\\

Part of the test-suite is verification of robustness to network partitions.
We show that network partitions through any of the phases of our synchronization protocol will leave the database in a consistent state.\\
This is partly due to our design choice of using a content-addressable store which never overrides existing data.
The entire synchronization process happens asynchronously and therefore does not block the database for writing by the local user.
Only once the data has been merged successfully, the node's head is updated.
Even the head update is asynchronous through the use of optimistic locking as described in Section \ref{sec:histo.protocol.merging}.
If the synchronization has to be cancelled at any stage, the node's head is simply not updated.
We therefore avoid the implementation of complex recovery logic.\\
Due to the separation of data propagation and local merging, the protocol can still efficiently operate during frequent network partitions.
The slow data propagation phase can continue where it stopped.
Only the comparatively fast local merging phase has to be redone when an error occurs during synchronization.

\section{Conclusion and Outlook}
By abstracting the complex process of synchronizing distributed data, Histo already drastically simplifies the implementation of collaborative applications.
We have developed a flexible architecture that supports extension of Histo and re-use of its components to test new approaches for history-based synchronization.\\
There still remains a lot of work to be done to make development of common collaboration scenarios even more productive.\\
Histo currently comes with support for a very generic model of hierarchical data as described in Section \ref{sec:histo.hierarchy}.
We exemplarically showed the mapping of a task manager's data model to Histo's internal data structure in Figure \ref{fig:histo.hierarchy}.
This could be taken further by exploring other use cases and finding common patterns in their data models.\\
We used the three-way merging algorithms for low-level data structures to implement higher-level ones as described in Section \ref{sec:histo.merging}.
This approach could be leveraged to build three-way merging support for other high-level data structures.
Interesting would be the application to specialized data structures like those of a spreadsheet application or a collaborative text-editor.\\
The goal is to have a broad range of synchronizable data structures that can be used as the foundation of collaborative applications.\\
To make Histo more accessible to developers, integration with client-side MVC frameworks could be built.
Popular web frameworks like Backbone.js \cite{backbone}, Ember.js \cite{ember} or AngularJS \cite{angular} include support for pluggable model stores.
