
\section{Handling Conflicts}
\label{sec:histo.conflicts}

Conflicts result in concurrently updating the same structure in different ways.
Section \ref{sec:histo.merging} has shown that in our scenario we have two types of conflicts:

\begin{itemize}
\item \emph{Value conflicts} happen if a structure's actual value is concurrently updated.
\item \emph{Position conflicts} result from the concurrent movement of the same structure within other structures.
\end{itemize}

An example for a value conflict is the concurrent update of the same key in a dictionary.
Position conflicts can appear in an ordered data structure like an ordered lists.
Even more complex position conflicts could result if we allow the movement of nodes within an entire tree.
Especially the combination of multiple concurrent move operations of multiple nodes can lead to non-trivial conflict scenarios.\\
In our data model we avoided these by not allowing the movement of task instances between projects.
The movement of a task to another project can therefore only be realized through a remove and insert operation effectively copying the data.
Insert and remove operations do usually not result in conflicts.
This an example on how a more constrained data model can limit the number of potential conflicts.\\
The resolution of conflicts is highly application dependent.
In Section \ref{sec:histo.merging} we have presented a hierarchical strategy for conflict resolution applied to instance models.
We were able to resolve conflicts identified in a high-level merge by applying a finer grained merge algorithm to the structures affected by the conflicts.\\
The remaining conflicts have to be resolved by the application developer.
The developer is thereby essentially expanding our merge hierarchy by adding another level of merging to resolve the conflicts.\\
This `application merge layer' does not necessarily have to work programatically - in many cases it cannot be.
A common solution is to develop a user interface, which allows the user to manually resolve the conflict.\\
In some applications the data is not that critical and conflicts are therefore simply being ignored.
In this case the application effectively `resolves' the conflict by randomly choosing one of the conflicting updates as the `correct' one.\\
Considering the example of conflicting but minor changes to a task title.
A user would most likely be annoyed everytime he is asked to manually resolve a title conflict.\\
A popular compromise is to randomly pick one version as the winner while still keeping the conflicting changes.
The user will not be interrupted in his workflow as conflicts occur while he still has the option to review and merge them.\\
Dropbox \cite{dropbox} follows this model by simply duplicating the file whenever a conflict occurs.
The file under the original name contains the picked `winner' of the conflict.
An additional file with a timestamp appended to the file name contains the changes of the `losing' node in the conflict.
An analog approach is chosen by Evernote \ref{evernote}, which duplicates notes on a conflict.\\
As we described in Section \ref{sec:couchdb}, CouchDB applies a similar model by deterministicially choosing one of the conflicting documents as the winner.
The deterministic winner picking is important to ensure that concurrent merge operations on different nodes pick the same document as a winner.\\
We believe that it adds value to our synchronization solution if we implement a winner picking model equivalent to CouchDB's.
It relieves the application developer from worrying about conflict resolution in early stages of the development cycle.
As long as the alternate outcomes of a conflict are still available we leave all application-specific merge options open.
This model therefore makes sure we are in line with our requirements for conflict handling, defined in Section \ref{sec:requirements.causality}, while still being convinient for user's of our framework.

