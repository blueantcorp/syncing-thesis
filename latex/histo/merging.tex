
\section{Reconciliation Through Three-Way Merging}
\label{sec:histo.merging}
In this section we will focus on three-way merging which is part of the reconciliation phase of our protocol developed in section \ref{sec:histo.protocol.detection-propagation}.\\
All necessary data for a merge has already been propagated from the source to the target node.
We have all information about the current state and the commit history of the source node available locally.
The current state is marked through the source tracking branch as described in section \ref{sec:histo.protocol.merging}.
The entire merging process can therefore happen offline without the need of any communication with the source node.\\
Our three-way merge algorithm is structured into three types of processes which are recursively executed on our data hierarchy:

\begin{itemize}
\item \textbf{Differencing}: we diff branch B and C with their common ancestor A.
\item \textbf{Merging}: we merge the two diffs A-B and A-C to a new diff result.
\item \textbf{Patching}: we apply the merged diffs as a patch to A which results in the merged state D.
\end{itemize}

Our current branch we refer to as B and the source tracking branch as C.
The common ancestor of B and C is defined as A.\\
Starting with a differencing phase we identify the changes made in the branches B and C since our common ancestor state A.\\
If follows a diff merging phase where the two diff results A-B and A-C are combined into one diff.
As its only input is the two diffs it does not require access to the actual states A, B or C.
In this phase conflicts can appearch if the two branches contain updates to the same parts of the state.\\
The merged diff can then be applied to the origin state A in order to create the actual merged state.
For this step we require a patch algorithm.\\
These three steps are actually applied to each hierarchy level or our state.
In the next section we give concrete examples of what the expected output of each level should be.
Based on the examples we will then look into implementation details of the respective algorithms.

\subsection{Merge Scenario}
\label{sec:histo.merging.scenario}
To show correctness of a merge algorithm we need to define sample model states with expected merge results.
This set of data can then be used as a test case for our implementation.
Based on an ancestor state all users start with, we define several possible branch states.\\
Merging of the branch states will happen at multiple levels of detail corresponding to our data hierarchy.
Each level will have its own difference and diff merge phase.
The result of the diff merge phases is then used to patch the common ancestor state.\\
The difference and merge algorithms that need to be applied will vary on each level depending on the data structures used.
The sample data is defined so that we can demonstrate each possible data structure that we mapped to in section \ref{sec:histo.hierarchy} (dictionaries, ordered dictionaries, ordered sets and ordered lists).\\

The root of our state is defined through an organization instance.
All other instances can be reached from it.
As defined in section \ref{sec:histo.hierarchy}, we differentiate between linking and embedding of instances.
Linking is done by only referencing an instance's ID, embedding is done by referencing its actual state hash along with the ID.\\

\begin{tabularx}{0.8\textwidth}{ l l X X }
\multicolumn{4}{ c }{\textbf{Ancestor state A of Organization}} \\
ID & Hash & Members (embedded) & Projects (embedded) \\
\hline
0
& O0A
& 1: U1A \newline 2: U2A \newline 3: U3A
& 4: P4A \newline 5: P5A \newline 6: P6A \newline 7: P7A
\end{tabularx} \\
\\

For completeness we include the state of embedded user instances - they will not be modified in this example:\\

\begin{tabularx}{0.3\textwidth}{ l l l }
\multicolumn{3}{ c }{\textbf{Ancestor state A of Users}} \\
ID & Hash & Name \\
\hline
1 & U1A & Rita \\
2 & U2A & Tom \\
3 & U3A & Allen
\end{tabularx} \\
\\

The projects embedded in the organization show the difference between linking (members) and embedding (projects):\\

\begin{tabularx}{\textwidth}{ l l l l X }
\multicolumn{5}{ c }{\textbf{Ancestor state A of Projects}} \\
ID & Hash & Project Name & Members (linked) & Tasks (embedded, ordered) \\
\hline
4 & P1A & Marketng Material & 1, 2, 3
& 8: T8A \newline 9: T9A \newline 10: T9A \newline 11: T11A
\\
5 & P5A & Product Roadmap & 1, 3 & 12: T12A
\\
6 & P6A & Staffing & 1 & 13: T13A \newline 14: T14A
\\
7 & P7A & Finances & 3 & 15: T15A
\end{tabularx} \\

We will leave out the details on the state of other entities (tasks and comments) as the instances shown so far already cover all required modeling aspects.\\

The state has been modified resulting in a new state in branch B:\\

\begin{tabularx}{0.8\textwidth}{ l l X X }
\multicolumn{4}{ c }{\textbf{State B of Organization}} \\
ID & Hash & Members (embedded) & Projects (embedded) \\
\hline
0
& O0AB
& 1: U1A \newline 2: U2A \newline 3: U3A
& 4: P4A \newline 5: P5A \newline 6: P6A \newline 7: P7A
\end{tabularx} \\
\\

\begin{tabularx}{\textwidth}{ l l l l X }
\multicolumn{5}{ c }{\textbf{State B of Projects}} \\
ID & Hash & Project Name & Members (linked) & Tasks (embedded, ordered) \\
\hline
4 & P1AB & Marketing Material & 1, 2
&
8: T8A \newline
11: T11A \newline
9: T9A \newline
10: T9A \newline
17: T17B
\\
5 & P5AB & Product Roadmap & 1, 3 & 12: T12A
\\
6 & P6A & Staffing & 1 & 13: T13A \newline 14: T14A
\\
7 & P7A & Finances & 3 & 15: T15A \newline 18: T18B
\\
16 & P16B & Sales Planning & 1, 2 & 
\end{tabularx} \\
\\

Through concurrent updates branch C emerged - its state is defined as the following:\\

\begin{tabularx}{0.8\textwidth}{ l l X X }
\multicolumn{4}{ c }{\textbf{Ancestor state C of Organization}} \\
ID & Hash & Members (embedded) & Projects (embedded) \\
\hline
0
& O0AC
& 1: U1A \newline 2: U2A \newline 3: U3A
& 4: P4AC \newline 5: P5AC
\end{tabularx} \\
\\

\begin{tabularx}{\textwidth}{ l l l l X }
\multicolumn{5}{ c }{\textbf{State C of Projects}} \\
ID & Hash & Project Name & Members (linked) & Tasks (embedded, ordered) \\
\hline
4 & P1AC & Marketing Strategy & 1, 2, 3
& 11: T11A \newline 8: T8A \newline 9: T9A \newline 10: T9A
\\
5 & P5AC & Product Strategy & 1, 3 & 12: T12A \newline 19: T19C
\end{tabularx} \\

We start the merging process with the root instance hashs:\\

\begin{tabular}{ l l l l l }
\multicolumn{5}{ c }{\textbf{Organization difference}} \\
ID & State A & State B & State C & Result \\
\hline
0 & O0A & O0AB & O0AC & conflict (concurrent update)
\end{tabular} \\

Both B and C have obviously modified the content of our organization in different ways resulting in a conflict.\\
We will try to resolve this conflict by differencing the data on a more detailed level.
The value of the `members' attribute has not been modified - we therefore look at the difference of the `projects' attribute: \\

\begin{tabular}{ l l l l l }
\multicolumn{5}{ c }{\textbf{`Projects' attribute difference of organization ID 0}} \\
ID & State A & State B & State C & Result \\
\hline
4 & P4A & P4AB & P4AC & conflict (concurrent update) \\
5 & P5A & P5AB & P5AC & conflict (concurrent update) \\
6 & P6A & P6A & - & remove \\
7 & P7A & P7AB & - & conflict (concurrent update and remove) \\
16 & - & P16B & - & insert
\end{tabular} \\
\\

Whenever the hash of an instance has only changed in one state, we carry the change over into the result.
If the hash has changed in both states, we have a conflict.
This merging phase will be implemented through difference and merge algorithms for dictionaries.\\
We will try to resolve conflicts by running a finer grained merge on the conflicting instance states.
ID 4 and ID 5 will therefore go into a lower level of merging while the changes in ID 6 and 16 are already accepted.
We cannot run a finer grained merge on ID 7 as it was removed in state C.
This conflict can therefore not be resolved and directly carried into the result.\\

Lets see if we can resolve conflicts 4 and 5 in the next merge.
At this level we will again look at individual attributes to find out which are actually affected by the updates.\\

\begin{tabularx}{\textwidth}{ l X X X l }
\multicolumn{5}{ c }{\textbf{Attribute difference of task ID 4}} \\
Attribute & State A Hash & State B Hash & State C Hash & Result \\
\hline
Project Name & Markting Material & Marketing Material & Marketing Strategy & conflict \\
Members & 1, 2, 3 & 1, 2 & 1, 2, 3 & update
\\
Tasks &
8: T8A \newline 9: T9A \newline 10: T9A \newline 11: T11A
&
8: T8A \newline
11: T11A \newline
9: T9A \newline
10: T9A \newline
17: T17B
& 11: T11A \newline 8: T8A \newline 9: T9A \newline 10: T9A
& conflict
\end{tabularx}\\
\\

At this level we can already carry over the update of the `Members' attribute in B to the result.
Note that if we had a conflict here, we would have run an ordered set difference and merge algorithm as it is shown later as part of the ordered dictionary algorithms.\\
The `Project Name' and `Tasks' attributes of B and C have been concurrently updated and are therefore in conflict.
This difference and merge step can again be realized with an algorithm for dictionaries (keys being attributes, values being attribute values).\\
If we define string values as atoms we have reached the most detailed level of merging for `Project Name' - the conflict will therefore be carried over into the result.
If we instead see strings as another substructure we can attempt to resolve the conflict by using a string merging algorithm.\\
We have mapped the embedded and ordered `Tasks' attribute to an ordered dictionary data structure.
We can therefore try to resolve the conflict by applying a merge algorithm for ordered dictionaries.\\

We will now describe the result we expect from merging the `Project Name' strings.
Note that all index positions in the following diffs are seen as relative to the ancestor state.
So even if one diff contains multiple insert or remove operations they are given as if they were all applied simultaneously to the ancestor state.\\

\begin{tabularx}{\textwidth}{ l X X }
\multicolumn{3}{ c }{\textbf{`Project Name' difference for project ID 4}} \\
Diff A-B & Diff A-C & Diffs Merged \\
\hline
insert `i' behind index 5 & insert `i' behind index 5 \newline remove from index 9 to 17 & insert `i' behind index 5 \newline remove from index 9 to 17
\end{tabularx}\\
\\

Applying this merge result to the ancestor state gives us an intuitive result: `Marketing Strategy'.\\

To merge the ordered dictionary structure representing the `Tasks' attribute we will apply two separate merge steps:

\begin{itemize}
\item Order merge: We can merge the order by using the keys of the ordered dictionary as an ordered set and applying an ordered set merge algorithm.
\item Value merge: The values are merged by applying an ordinary dictionary merge algorithm. 
\end{itemize}

The expected output of the order merge for the `Tasks' attribute:\\

\begin{tabularx}{\textwidth}{ X X X }
\multicolumn{3}{ c }{\textbf{`Tasks' order difference for project ID 4}} \\
Diff A-B & Diff A-C & Diffs Merged \\
\hline
move index 3 behind index 0 \newline insert 6 behind index 3
& move index 3 behind index -1
& insert 6 behind index 3 \newline
conflict (`move index 3 behind index 0' and `move index 3 behind index -1')
\end{tabularx}\\
\\

We have been able to carry over one operation into the result while we still have an update conflict.
There is no way we can do an even finer grained merge to resolve this conflict.
The application developer will have to implement a custom merge solution - possibly even asking the user which operation to choose.\\
Depending on how the conflict is resolved, possible results are:

\begin{itemize}
\item 8, 11, 9, 10, 17
\item 11, 8, 9, 10, 17
\end{itemize}

The value merge works analog to the merge we did for the `Projects' attribute in the organization instance.
We will skip this phase as none of the tasks have been updated.\\

The merge of ID 5 works analogous - we will therefore skip the attribute merging step and jump right into the string merge:\\

\begin{tabularx}{\textwidth}{ X X X }
\multicolumn{3}{ c }{\textbf{`Project Name' difference of project ID 5}} \\
Diff A-B & Diff A-C & Diffs Merged \\
\hline
remove from index 8 to 15 \newline insert `Planning' behind index 7
& remove from index 8 to 15 \newline insert `Strategy' behind index 7
& remove from index 8 to 15 \newline insert `Planning' behind index 7
\newline insert `Strategy' behind index 7
\end{tabularx}\\
\\

We have applied the same merge logic that helped us resolve the name conflict in project ID 4.
But the result we get here is not what a user would expect.
Applying the merged operations to the ancestor state results in: `Product StrategyPlanning'.\\
This is a good example for the violation of intention preservation.
Both in state B and C the user actually intended to \emph{replace} the word `Roadmap'.
Concurrently replacing the same word should clearly cause a conflict.
Our differencing algorithm has no notion of a `replace' operation.
All it sees is a remove of `Roadmap' and an insert of some other word.\\
This example shoes the limits of generic merging algorithms that preserve user intention.\\
In practice it would propably be more suitable to define the `Project Name' as an atom.
As names are short it is more likely that an update is actually intended as a replace operation of the entire string.

\subsection{Differencing}
\label{sec:histo.merging.diff}
Differencing algorithms have been studied extensively.
There exist efficient solutions for a range of data structures.
It is not a focus of our thesis to develop the most efficient differencing algorithm matching our scenario.
Our goal in this section is to show the practical feasibility of the three-way merging component in our architecture.
We favor a simple solution, re-using existing concepts so that we can expand our focus to other areas of our synchronization protocol.\\

\emph{Ordered list} or string data structures have been the main focus in previous research on difference algorithms.
Myers presented an efficient algorithm with $\mathcal O(n*d) $ time and space complexity to difference two strings A and B with
$ n $ representing the sum of the lengths of two strings and $ d $ the size of the shortest edit script transforming A to B \cite{Myers:1986wi}.
The shortest edit script is equivalent to the result of a differencing algorithm.
As in practical applications differences are usually small the algorithm performs well.\\

Given the ordered list difference algorithm we can re-use it to build an algorithm for \emph{ordered sets}.
Ordered lists can only have differences in the form of \emph{insert} or \emph{remove} operations.
Ordered sets extend this - the simultaneous remove and insert of a globally unique element is now considered as a move operation.\\
To implement an ordered set algorithm we can therefore take the output of the ordered list algorithm and scan the result for the remove and insert of the same element.
This can be efficiently implemented through a hash:

\begin{enumerate}
\item Scan the diff result and build a hash for all removed elements.
\item Scan the result again and test for all inserted elements whether they are included in the hash.
\item If a match is found replace the remove and insert operations through a single move operation in the result.
\end{enumerate}

The time complexity of building a hash can be estimated with $ \mathcal O(n * log(n)) $ with $ n $ representing set size.
The match searching has linear time complexity.
We therefore only add $ \mathcal O(n * log(n)) $ complexity to Myers difference algorithm for a naive solution for ordered sets.\\

For \emph{ordinary sets} we can implement a simple solution through two hashs of the respective set entries combined with a scan through each set:

\begin{enumerate}
\item Add all entries of set A to a hash $ H_A $ and those of set B to a different hash $ H_B $.
\item Scan set A and test for matches in hash $ H_B $.
\item If no match is found add a remove operation to the result.
\item Scan set B and test for matches in hash $ H_A $.
\item If no match is found add an insert operation to the result.
\end{enumerate}

\emph{Insert} and \emph{remove} are the only operations in a set.\\
The time complexity for building each hash is again $ \mathcal O(n * log(n)) $.
Searching for matches has linear complexity which results in a time complexity $ \mathcal O(n * log(n)) $ for the entire algorithm.\\

An \emph{ordinary dictionary} data structure has \emph{insert}, \emph{remove} and \emph{update} operations.
If the instance described through the dictionary has a fixed set of attributes it would actually only need to support an update operation.
In modern web applications it is not uncommon that there is no fixed data schema.
It is often the case that new attributes are added to instances at runtime.
Even if there is a fixed schema it might be changed through a software update with old instances not being migrated to the new schema.
We should therefore support the full set of dictionary operations in our difference algorithm.

A simple and efficient solution is to again use two hashes for fast lookup combined with a scan through both dictionaries:

\begin{enumerate}
\item All all keys and values of dictionary A to hash $ H_A $ and those of dictionary B to hash $ H_B $.
\item Scan through all key-value entries of A.
\item If the key is not included in hash $ H_B $, add a remove operation to the result.
\item If the key is included and the value in $ H_B $ is different, add an update operation.
\item Scan through all key-value entries of B.
\item If the key is not included in hash $ H_A $, add an insert operation to the result.
\end{enumerate}

Building the hash is again estimated with time complexity $ \mathcal O(n * log(n)) $, scanning both dictionaries has linear time complexity. The combined time complexity is therefore again $ \mathcal O(n * log(n)) $.\\
Depending on the application, its instances are often already implemented with hash-like lookup performance - in this case we could skip step 1.
The time complexity is in this case only linear.\\

As already briefly described, \emph{Ordered dictionary} differencing can be implemented as a combination of difference algorithms for dictionaries and for ordered sets.\\
In addition to ordinary dictionaries, ordered dictionaries have a \emph{move} operation which supports changing of an entry's position.\\
The changes caused by \emph{insert}, \emph{update} and \emph{remove} operations can be identified by running an ordinary dictionary difference algorithm.\\
To track the movement of entries we represent all keys in the ordered dictionary as an ordered set.
We can then re-use our ordered set difference algorithm to identify move operations.
The output of the ordered-set algorithm will include \emph{insert} and \emph{remove} operations as well.
As those have already been identified they are simply ignored and not added to the result.\\

\subsection{Diff Merging}
\label{sec:histo.merging.merge}

We will now look into strategies for merging the diff results of the algorithms described in the previous section.
Our focus lies on the identification of potential conflicts based on three-way merging semantics.\\

\emph{Ordered lists} only support insert and remove operations.
Concurrent insert or remove operations are never conflicting as they do not update existing structures.
As we have seen in the example at the end of section \ref{sec:histo.merging.scenario} this is not necessarily in line with a user's intentions.
When editing text users often intend to \emph{replace} content although their actions are represented through a remove and insert operation.
This is why most concurrent version control systems have the notion of \emph{areas} in text.
Concurrent insert or remove operations in overlapping areas are considered as update operations to the same content and therefore result in conflicts.\\
The optimal definition of an area can vary depending on the application.
It may be the entire edited paragraph, a number of lines, sentences, words or even characters.\\
When modeled as an ordered list we can only vary the number of list elements defining the size of an area.\\
Concurrent insert or remove operations which are not in overlapping areas are simply both carried over to the merge result.\\

\emph{Ordinary sets} have the same operations as ordered lists except that each element is unique and we now have no element order.
Without a specified order we can not define \emph{areas}.
Merging two diffs is therefore trivial as no conflicts can occur.
We simply combine the set of operations of both diffs into one large diff.\\

\emph{Ordered sets} add move operations as we can uniquely identify each element.
Move operations can lead to conflicts if the same element is concurrently moved to different locations.
If move operations are not conflicting they are carried over into the merge result.
Insert and remove operations are merged the same way as defined for ordered lists.\\

\emph{Ordinary Dictionaries} have insert, remove and update operations.
Concurrent insert and remove operations never lead to conflicts - they are carried over into the merge result.
Update operations can lead to conflicts if the value of the same key is concurrently modified.\\

\emph{Ordered Dictionary} diff merging is realized through the combination of merge algorithms for ordered sets and ordinary dictionaries.
As with the respective difference algorithm we use the ordered set merge algorithm only for merging the entry positions.
Insert, update and remove operations are merged using the ordinary dictionary merge algorithm.

\subsection{Patching}
\label{sec:histo.merging.patching}

Patching is defined as an algorithm applying the identified differences between state A and B as operations to state A.
A computed diff A-B patched to A will result in B.
For our three-way merging algorithm, patching constitutes the last phase. 
Two merged diffs are applied as a patch to the common ancestor state resulting in the actual merged state.\\
We again need patching algorithms for each data structure we intend to support.
The implementation needs to consider that simply executing the operations defined in the diff can lead to a wrong result.
This is true if a change operation has side effects on other parts of the state.\\
In our set of data structures this is only the case for ordered lists and ordered sets.
Applying an insert operation at the specified index, renders the indexes of following operations wrong.
A simple example:

\begin{itemize}
\item State A: `Marketng Matrial'
\item Diff:\\
insert `i' behind index 5\\
insert `e' behind index 11
\end{itemize}

Applying the first diff operation to A results in: `Marketing Matrial'.\\
If we now execute the second operation at the specified index 11 we get: `Marketing Maetrial'.\\
The implementation therefore needs to treat the given indexes as relative positions in the ancestor state.\\
The diff operations of sets and dictionaries can be executed as they are without taking side-effects into consideration.

\subsection{Hierarchical Merging}
As we have seen in our scenario, merging of our structured data is actually a hierachical process combining all of the described algorithms.
Our strategy is to start at the highest-level representing sub-structures using their cryptographic hash.
Whenever we identify an update on the hash of a sub-structure we do a finer grained diff and merge on the structure itself:

\begin{enumerate}
\item Represent all instances of an entity as a dictionary with instance IDs as keys and a cryptographic hash of their attribute values as dictionary values.
\item Use the dictionary diff algorithm to identify inserted, removed and updated instances on A-B and A-C.
\item Merge the dictionary diffs to result \emph{M1}.
\item Remove each instance with an update conflict from \emph{M1}.\\
Represent it as a dictionary with keys being attribute names and values being a cryptographic hash of attribute values.
\item Use the dictionary diff algorithm to identify inserted, removed and updated attributes.
\item Merge the per-instance diffs to result \emph{M2}.
\item Remove each non-atomic attribute with an update conflict from \emph{M3}.\\
Run the respective algorithm:\\
Attributes with (ordered) sets of instance IDs: (ordered) set diff algorithm.\\
Attributes with string values: ordered list diff algorithm.
\item Merge the attribute diffs to result \emph{M3}.
\item Resolve all remaining conflicts in \emph{M1}, \emph{M2} and \emph{M3} manually.
\item Patch the respective substructures of state A using \emph{M1}, \emph{M2} and \emph{M3}.
\end{enumerate}

This approach can not only be applied to difference entity-relationship models, it can actually be adapted for any type of data with a hierarchical structure.
