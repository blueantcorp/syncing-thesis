% This template was initially provided by Dulip Withanage.
% Modifications for the database systems research group
% were made by Conny Junghans and Jannik Strötgen.

\documentclass[
     12pt,         % font size
     a4paper,      % paper format
     BCOR10mm,     % binding correction
     DIV14,        % stripe size for margin calculation
     liststotoc,   % table listing in toc
     bibtotoc,     % bibliography in toc
     idxtotoc,     % index in toc
%     parskip       % paragraph skip instad of paragraph indent
     ]{scrreprt}

%%%%%%%%%%%%%%%%%%%%%%%%%%%%%%%%%%%%%%%%%%%%%%%%%%%%%%%%%%%%

% PACKAGES:

% Use German :
\usepackage[USenglish]{babel}
% Input encoding
\usepackage[utf8]{inputenc}
% Font encoding
\usepackage[T1]{fontenc}
% Index-generation
\usepackage{makeidx}
% Einbinden von URLs:
\usepackage{url}
% Special \LaTex symbols (e.g. \BibTeX):
\usepackage{doc}
% Include Graphic-files:
\usepackage{graphicx}
% Include doc++ generated tex-files:
%\usepackage{docxx}
% Include PDF links
%\usepackage[pdftex, bookmarks=true]{hyperref}

% Fuer anderthalbzeiligen Textsatz
\usepackage{setspace}

\usepackage{listings}
\usepackage{color}

\definecolor{green}{rgb}{0,0.6,0}
\definecolor{gray}{rgb}{0.5,0.5,0.5}
\definecolor{mauve}{rgb}{0.58,0,0.82}

\lstset{
  belowcaptionskip=1\baselineskip,
  captionpos=b,
  breaklines=true,
  frame=L,
  xleftmargin=\parindent,
  showstringspaces=false,
  basicstyle=\footnotesize\ttfamily,
  commentstyle=\itshape\color{gray},
  numbers=left
}

% hyperrefs in the documents
\usepackage[bookmarks=true,colorlinks,pdfpagelabels,pdfstartview = FitH,bookmarksopen = true,bookmarksnumbered = true,linkcolor = black,plainpages = false,hypertexnames = false,citecolor = black,urlcolor=black]{hyperref} 
%\usepackage{hyperref}

% force graphics to be in same section as declared
\usepackage[section]{placeins}

\setlength{\parindent}{0cm}

\usepackage{tabularx}
\usepackage{import}
%%%%%%%%%%%%%%%%%%%%%%%%%%%%%%%%%%%%%%%%%%%%%%%%%%%%%%%%%%%%

% OTHER SETTINGS:

% Pagestyle:
\pagestyle{headings}

% Choose language
%\selectlanguage{english}
%\newcommand{\setlang}[0]{\selectlanguage{USenglish}\nonfrenchspacing}


\begin{document}

% TITLE:
\pagenumbering{roman} 
\begin{titlepage}


\vspace*{1cm}
\begin{center}
\vspace*{3cm}
\textbf{ 
\Large Ruprecht Karls University Heidelberg\\
\smallskip
\Large Institute of Computer Science\\
\smallskip
\Large Database Systems Research Group\\
\smallskip
}

\vspace{3cm}

\textbf{\large Bachelor Thesis} % Studienarbeit, Interdisziplinaeres Projekt

\vspace{0.5\baselineskip}
{\huge
Offline Usage and Synchronization in Mobile Apps with HTML5
}
\end{center}

\vfill 

{\large
\begin{tabular}[l]{ll}
Name: & Mirko Kiefer\\
%Matricle: & 2746040\\
%Supervisor: & Prof. Dr. Gertz\\
%Date of Submission: & \today
\end{tabular}
}

\end{titlepage}

\onehalfspacing

\thispagestyle{empty}

\vspace*{100pt}
I declare that this thesis was composed by myself and that the work contained therein is my
own, except where explicitly stated otherwise in the text.

\vspace*{50pt}


Date of Submission: \today
\newpage

% Add a brief summary of your topic and contributions (Zusammenfassung) in German and in English:
\chapter*{Abstract}

% This file contains an abstract of your thesis, with approximaltely 300-500 words

Inspired by distributed version control for source code we develop Histo, which is a framework for peer-to-peer data synchronization in mobile applications.\\
Our objectives when developing Histo are driven by exemplaric workflows in a collaborative task manager.\\
We see \emph{offline support} as a key requirement of modern mobile applications.
Histo is designed to keep all data locally on the client device to ensure a user is not blocked from using an application.\\
Every application has a different data model, we develop methods to support \emph{flexible data models}.
We show that by mapping an application's data to a hierarchical model, we can efficiently synchronize data.\\
Locking mechanisms are shown to be non-feasible when working with loosely connected devices.
The \emph{synchronization protocol}, which is at the heart of Histo, is therefore designed to work without any locking logic.
We instead rely on a concept of \emph{optimistic synchronization}, which guarantees \emph{eventual consistency} of an application's data.\\
Handling concurrent edits and resulting conflicts correctly is another core element of Histo.
Histo merges concurrent edits and identifies conflicts using \emph{three-way-merging}.
We find that the only way to implement three-way-merging in a distributed setting is by tracking the edit history on each device.\\
Histo's synchronization protocol is shown to perform well in a range of network topologies.
The most extreme cases being \emph{client-server} and \emph{peer-to-peer}.
For each network topology, we show how assumptions we make on the data history allows to minimize the amount of data stored on each device.\\
We focus on developing a practical solution that works on a broad range of devices.
Histo is therefore implemented with \emph{open web standards}.
To our knowledge there are no publicly available solutions with these objectives.\\
We see potential to continue the work by exploring other application scenarios and adding support for different types of data.
This could include conflict handling support for specialized data structures like those of a spreadsheet or a text editor.
\newpage

\chapter*{Zusammenfassung}

% This file contains the German version of your abstract, with about 300-500 words


\newpage

% MAIN PART:
% Table of contents (Inhaltsveryeichnis)
\tableofcontents
\cleardoublepage
\pagenumbering{arabic} 

% List of figures (Abbildungsverzeichnis):
%\listoffigures
% List of tables (Tabellenverzeichnis):
%\listoftables

%%%%%%%%%%%%%%%%%%%%%%%%%%%%%%%%%%%%%%%%%%%%%%%%%%%%%%%%%%%%%%%
% Here, the actual content of your thesis begins
% You can either put all the text here or use individual files to store the chapters of your thesis.
% Below are templates for both alternatives.

\subimport*{introduction/}{index}

\subimport*{background/}{index}

\subimport*{main/}{index}

\subimport*{realization/}{index}

% Alternative: put content in separate files
% Check the difference between including these files using \input{filename} and \include{filename} and see which one you like better
%\chapter{Einleitung}\label{intro}
%
\chapter{Introduction}
\label{sec:intro}

\section{Motivation}
Applications that allow users to collaborate on data on a central server are in widespread use.
Popular examples are document authoring tools like Google Docs, project collaboration apps like Basecamp or Trello or even large scale collaboration projects like Wikipedia.\\

The traditional architecture of collaborative applications follows a client-server model where the server hosts the entire application logic and persistence.
Users access the application through a thin client, most commonly a web browser.
The browser only has to display user interfaces that are pre-rendered by the server.\\
This model works well when using desktop computers with a realiable, high-speed connection to the server.\\

Rising expectations on the user experience drove developers to increasingly move application logic to the client.
Initially this has only been the logic required to render user interfaces.
The server still hosted most of the application logic to pre-compute all relevant data for the client.\\
Moving the interface rendering to the client reduces the amount of data that has to be transferred and makes the application behave more responsive.\\

The widespread adoption of mobile devices forces developers to re-think their architecture again.
Users can now carry their devices with them and expect their applications to work outside their home or office network.
Applications therefore have to work with limited mobile Internet access or often no access at all.\\
The only way to support this is by moving more of the application logic to the client and by replicating data for offline use.
The clients are now not only responsible for rendering interfaces but also implement most of the application logic themselves.\\
The new architecture comes at a high price - the additional client logic and persistence adds a lot of complexity.
While in the server-centric model developers only had to maintain a single technology set, they now face different technologies on each platform they aim to support with a fat client.\\
The ability to use the application offline requires an entire new layer of application logic to manage the propagation and merging of changes and to resolve conflicts.
The only responsiblity of the server in this model is the propagation of data between clients.\\

Most users today carry a notebook, a smartphone and maybe even a tablet computer with them.
They often want to work with the same data on different devices.
Apps need to support workflows like adding some items to a Todo-Manager on a notebook and subsequently reviewing them on a smartphone.
This implies that even simple applications that are meant for single-users have to aquire collaborative features.
A single-user with multiple devices is from a technical perspective effectively collaborating with itself.\\
Today's applications only achieve this through  data synchronization between the devices and a central server.
If the user is mobile and does not have a reliable Internet connection he is stuck with outdated data on his smartphone.
This problem can only be resolved by supporting the direct synchronization between devices.
The clients can now basically act as servers themselves and manage propagation of data to other clients.\\
The actual server does not have to disappear in this model.
But like the clients it is just another node on the network.
The difference is that the server node is continuously connected to the Internet and can therefore play a useful role as a fallback.\\
Note that this only describes the most extreme scenario - in most real-world applications we will see a hybrid-architecture where clients can synchronize most data directly but the server still manages security or enforces other constraints.\\
Building such a distributed data synchronization engine including all relevant aspects is very complex and beyond the reach of a small team of app developers.
It is also way beyond the scope of this thesis.
As described in the next section we will focus on a set of problem statements and use cases.

TODO:

- add Things app story on how hard it is

- add graphics

\section{Objectives and Approach}

This thesis aims to develop patterns and tools to make the development of offline capable, collaborative apps more productive.\\

The guiding questions are:
\begin{itemize}
\item \emph{Offline Availability}: How can we enable the operation of a collaborative app with frequent network partition?
\item \emph{Synchronization Protocol}: How can we efficiently synchronize changed data directly between unreliably connected devices?
\item \emph{Application Integration}: How can we abstract the synchronization logic to be as unintrusive as possible to an application?\\
\end{itemize}

A collaborative app that has to function with unreliable network connection implies that we can not rely on the traditional thin client model.
We have to think about ways to make both data and logic available offline.\\

Being able to synchronize data directly between devices forces us to develop a distributed architecture.\\

Efficient synchronization means that we aim to minimize the amount of redundant data sent between devices.
We have to figure out ways to identify changes in the data.\\

Combined with the requirement to be unintrusive we exclude solutions that require the application to explicitly track changes in the code.
The identification of data changes should be decoupled from the main application logic.
This ensures that an upgrade of traditional applications requires minimal effort.\\

We will refine this set of requirements by breaking down common use cases and evaluating existing solutions that support offline-capable applications.\\

Important questions which are out of scope of this thesis are:

\begin{itemize}
\item \emph{Security}: How can we manage access rights and encryption in a distributed architecture?
\item \emph{Device Discovery}: How can we discover devices in a network to collaborate with?
\item \emph{Data Transmission}: How is the data propagated among devices on a technical level?
\end{itemize}

TODO:

- phrase problem statements - not only questions

- what is unique to our approach? (focus on practical realization with open web standards and modern tools, offline as default)

\section{Structure of the thesis}
Here you describe the structure of the thesis. For example:

In Kapitel~\ref{background} werden grundlegende Methoden für diese Arbeit vorgestellt.
%
%\chapter{Voraussetzungen}\label{bg}
%
\chapter{Background}\label{background}

Here you discuss some basics for your work and outline existing research in the area of your thesis by citing research papers like~\cite{Lindholm:2009wo} by Lindholm and~\cite{DeCandia:2007ui,Ratner:2001wz} Candia.

\section{Edit-Based Synchronization}
- Operational Transformation
- Commutative Replicated Data Types

\section{State-Based Synchronization}
- async through diff computation
- easier to integrate
- Fraser2009 differential sync

\section{Three-Way Merging}
- 3DM tool (Lindholm)

\section{Most Recent Common Ancestor}

\section{HTML5 and Offline Storage}


% References (Literaturverzeichnis):
% a) Style (with abbreviations: use alpha):
\bibliographystyle{unsrt}
% b) The File:
\bibliography{references,manual_references}

\end{document}
